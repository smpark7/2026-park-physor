\documentclass[letterpaper]{physor2026}

%%% Packages Required by Class (already included)
% fancyhdr
% lastpage
% titling
% titlesec
% ragged2e
% enumitem
% amsmath
% graphicx
% geometry
% newtxtext
% newtxmath
% hyperref
% cleveref
% caption
% authblk
% apptools
% appendix
% ifpdf
% epstopdf

%%% Some other useful packages
% \usepackage{tikz}
% \usepackage{color}
\usepackage{subcaption}
% \usepackage{algcompatible}
% \usepackage{bm}
% \usepackage{array}

\usepackage{siunitx}
\graphicspath{{images/}}
\usepackage[acronym]{glossaries}
\usepackage{nomencl} % If needed
\makenomenclature


\title{A Multiscale 3-D/1-D Multiphysics Molten Salt Reactor Simulation Framework Based on
Moltres-SAM Coupling}

%%% Authors (use arabic numbers: 1, 2, 3, etc. for affiliationNumber)
%%% \addAuthor{GivenName MiddleInitial. FamilyName}{affiliationNumber}
\addAuthor[sunmyung.park@austin.utexas.edu]{Sun Myung Park}{1}
\addAuthor{Cole A.\ Gentry}{1}
\addAuthor{Nicholas Luciano}{1}
\addAuthor{Kevin T.\ Clarno}{1}
\addAuthor{Rok Krpan}{2}
\addAuthor{Carlo Fiorina}{2}
\addAuthor{Jean Ragusa}{2}

%%% Affiliations (from authblk)
%%% \addAffiliation{affiliationNumber}{Name of Institute, City, State/Country}
\addAffiliation{1}{The University of Texas at Austin, Austin, TX, USA}
\addAffiliation{2}{Texas A\&M University, College Station, TX, USA}

%%% Write text for abstract
%%% Most text modifying commands will work in abstract
\Abstract{%
Reactor analysis tools for molten salt reactors (MSRs) are critical for supporting their
development and commercial deployment. However, the liquid fuel form in MSRs poses computational
challenges such as strong temperature reactivity feedback and delayed neutron precursor drift.
This work introduces a multiscale 3-D/1-D multiphysics simulation framework for molten salt reactor
analysis by coupling two MOOSE-based codes, Moltres and SAM, for neutronics and thermal-hydraulics,
respectively. This new multiphysics tool aims to provide a means for engineering-level MSR analyses
while retaining 3-D spatial resolution of the core for accurate reactivity feedback and precursor
tracking. We present the coupling scheme between Moltres and SAM implemented entirely with native
coupling interfaces provided by MOOSE. Thereafter, we demonstrate the multiphysics tool on a
full-core MSR model derived from the Molten Salt Reactor Experiment (MSRE). The steady-state
neutronics and thermal-hydraulics results were consistent with expected behavior of
a graphite-moderated MSR. Ongoing work focuses on improving model fidelity and computational
performance and verifying results against other multiphysics tools.
}


%%% List up to 5 keywords separated by a comma
\keywords{Molten salt reactor, multiphysics, multiscale, MOOSE}

%%% Provide a short title for the header on odd pages
\shortTitle{A Multiscale 3-D/1-D Multiphysics Molten Salt Reactor Simulation Framework}

%%% Provide a short author listing for the header on even pages
\authorHead{S. P, et al.}

%%% If LaTeX reports the line number of an error at \begin{document} it
%%%   is most likely due to an error in one of the commands above
\begin{document}
\include{acros}

\section{INTRODUCTION}\label{sec:introduction}

\glspl{MSR} have gained renewed interest as a promising Generation IV reactor concept due to their
strong negative reactivity feedbacks, high operating temperature, and potential for online fuel
reprocessing. Unlike solid-fuel reactors, \glspl{MSR} feature mobile fuel inventory, and the
delayed neutron precursors drift with the circulating fuel salt. Therefore, the reactor kinetics
are tightly coupled with thermal-hydraulics.

Simple 1-D flow models that combine multiple fuel salt channels into fewer 1-D channels have proved
to be accurate at reproducing integral reactor responses (e.g., $k_{\text{eff}}$), but these models
cannot model transient scenarios with significant spatial shifts in temperature and neutron flux.
On the other hand, high-fidelity 3-D flow models remain too computationally intensive for
engineering-level tasks.

In this work, we develop a multiscale 3-D/1-D multiphysics \gls{MSR} model by coupling two
\gls{MOOSE}-based applications, Moltres \cite{lindsay_introduction_2018, park_verification_2022}
(3-D \gls{MSR} multiphysics code) and \gls{SAM} \cite{hu_sam_2025} (modern
nuclear system code for advanced reactor safety analysis). Moltres provides 3-D neutron diffusion,
precursor source, and graphite heat conduction modeling while \gls{SAM} provides 1-D multichannel
flow modeling for system thermal-hydraulics and precursor drift to model all salt channels in a
full-core \gls{MSR} geometry and ex-core segments of the primary loop. The 1-D multichannel flow
model reduces the computational cost relative to 3-D flow models while simultaneously retaining
more 3-D spatial resolution relative to single-/few-channel 1-D flow models. The long-term
objective of this research is to develop an intermediate-fidelity \gls{MSR} simulation framework to
support the digital twin development efforts under the Digital Molten Salt Reactor Initiative at
The University of Texas at Austin.

\section{SOFTWARE DESCRIPTION}

\subsection{MOLTRES}

Moltres is an open-source multiphysics reactor simulation software developed for \gls{MSR}
modeling. Moltres solves the neutron
diffusion or $S_N$ neutron transport equations for neutronics modeling. The delayed neutron
precursor model in Moltres includes advection capabilities that can natively couple to any flow
model in \gls{MOOSE} such as the Navier-Stokes module or the \gls{SAM} nuclear system code.

Moltres has been used in past work to model the \gls{MSRE}
\cite{lindsay_introduction_2018, park_advancements_2025}, \gls{MSFR} \cite{park_advancement_2020},
and fast-spectrum chloride \glspl{MSR} \cite{lawson_development_2024}. We chose Moltres for this
work due to its \gls{MSR} modeling capabilities and its integration within the \gls{MOOSE}
ecosystem. As a \gls{MOOSE}-based tool, Moltres benefits from highly scalable mesh and
solver routines available through \gls{MOOSE} and natively couples to \gls{SAM} (also a
\gls{MOOSE}-based tool). Moltres' open-source nature accelerates the development and testing of new
features to improve \gls{MSR} modeling and simulation performance.

\subsection{SYSTEM ANALYSIS MODULE (SAM)}

\gls{SAM} is a modern system analysis tool for advanced reactor design optimization, safety
analyses, and licensing support. Though originally developed for sodium fast reactors, \gls{SAM}
developers have implemented and tested numerous advanced features tailored for salt-cooled reactors
such as multiphase species transport, freeze plug modeling, and drift flux modeling. In most
applications, \gls{SAM} solves the 1-D incompressible but thermally expandable flow equations along
with heat transfer equations. \gls{SAM} also supports coarse-mesh, multi-D flow modeling typically
used for reactor designs with large coolant pools or porous medium flow in pebble-bed reactors.
This work uses \gls{SAM}-Lite, a non-export controlled version of \gls{SAM} with some restricted
capabilities removed. Since all capabilities demonstrated in this work can be replicated using
\gls{SAM}, we drop the \textit{-Lite} suffix for brevity.

In general, two different \gls{SAM}-based \gls{MSRE} models have been demonstrated in the current
literature. The first \gls{SAM}-based model approximates the \gls{MSRE} core as a single 1-D pipe
while preserving the height and flow area \cite{hu_fy21_2021}. This model has been validated
against experimental data
from various transient experiments by accurately reproducing integral neutronic and temperature
metrics. The second \gls{SAM}-based model couples the Griffin reactor physics code to \gls{SAM} and
models the \gls{MSRE} core as a homogenized, 2-D axisymmetric region
\cite{jaradat_multiphysics_2023}. Griffin solves the neutron diffusion
equations and \gls{SAM} solves porous media flow equations in the 2-D core region. In both
\gls{SAM}-based models, \gls{SAM} models the ex-core salt loop as several serially connected 1-D
pipe regions forming a loop with the core region.

\section{MODEL DESCRIPTION} \label{sec:model}

%Figure \ref{fig:msre} shows a vertical
%cross section view of the \gls{MSRE} geometry.

%\begin{figure}[h]
%    \centering
%    \includegraphics[width=0.5\columnwidth]{msre-picture}
%    \caption{Vertical cross section of the \gls{MSRE} vessel \cite{robertson_msre_1965}.}
%    \label{fig:msre}
%\end{figure}

The reference geometry for this work is from the \gls{MSRE} built in 1964 at Oak Ridge National
Laboratory \cite{robertson_msre_1965}. It was a graphite-moderated \gls{MSR} that initially
operated on $^{235}$U and later replaced with $^{233}$U.
The \gls{MSR} geometry in this work incorporates most of the \gls{MSRE} design as a
representative model of graphite-moderated \glspl{MSR}. Several geometric simplifications have been
made to facilitate model creation for code-to-code verification between multiphysics \gls{MSR}
analysis tools of varying model fidelity. This verification exercise is ongoing and
formalized as a series of \gls{MSR} progression problem benchmarks
\cite{gentry_graphite-moderated_2025}.

\begin{figure}[!b]
    \centering

    \begin{subfigure}[b]{0.58\textwidth}
      \centering
      \includegraphics[width=\textwidth]{cross-section}
      \caption{Horizontal cross section}
      \label{fig:cross}
    \end{subfigure}
    \hfill
    \begin{subfigure}[b]{0.38\textwidth}
      \centering
      \includegraphics[width=\textwidth]{vertical}
      \caption{Vertical cross section}
      \label{fig:vertical}
    \end{subfigure}
    
    \caption{Horizontal and vertical cross sections of the 3-D \gls{MSR} core model.}
    \label{fig:geometry}
\end{figure}

Figure \ref{fig:geometry} shows the horizontal and vertical cross section views of the 3-D \gls{MSR}
core geometry. The reactor geometry consists of 1136 regular fuel channels and four central,
annular salt channels each encircling four control rod guide thimbles. The fuel channel centroids
form a 0.0508 m$\times$0.0508 m square grid in the graphite matrix. We simplified the
peripheral core region by having the graphite core terminate with a 2.54 cm-thick vessel wall. The
fuel-graphite core region is sandwiched by the upper and lower plena regions filled with 100\% fuel
salt and surrounded radially by the same vessel wall. Table \ref{table:dimensions} lists some of
the key physical dimensions of the \gls{MSR} core model.

\begin{table}[htb]
  \centering
  \caption{Key physical dimensions of the \gls{MSR} core model.}
  \label{table:dimensions}
  \begin{tabular}{l S}
    \toprule
    Parameter & {Value} \\
    \midrule
    Lower plenum height [m] & 0.1875 \\
    Graphite height [m] & 1.70027 \\
    Upper plenum height [m] & 0.254 \\
    Core radius [m] & 0.7366 \\
    Central channel inner radius [m] & 0.0254 \\
    Central channel outer radius [m] & 0.031992 \\
    \bottomrule
  \end{tabular}
\end{table}

Salt flows upwards through the core, leaves the core into 1-D pipes connecting to the pump,
followed by the heat exchanger, before flowing through the downcomer and re-entering the core from
the bottom. Out-of-core components were adapted from an existing \gls{SAM} \gls{MSRE} model
\cite{hu_fy21_2021} openly available in the Virtual Test Bed Repository
\cite{giudicelli_virtual_2023}. The overall geometry is a preliminary \gls{MSR} geometry created
for this work and is subject to future changes. We plan to include control rod components in
future publications.

\begin{figure}[!htbp]
    \centering
    \includegraphics[width=0.8\columnwidth]{coupling}
    \caption{Moltres and \gls{SAM} MultiApp coupling setup and data transfers.}
    \label{fig:coupling}
\end{figure}

We coupled Moltres and \gls{SAM} as \textit{sibling subapps} using the MOOSE MultiApp system. A
dummy \textit{main app} does not perform any calculations but dictates all data transfers between
the two subapps.
Figure \ref{fig:coupling} shows an overview of how Moltres and \gls{SAM} are
coupled and the data transfers involved between the two codes. After solving the
3-D two-group neutron diffusion and graphite heat conduction equations, Moltres transfers neutron
source distribution for precursor source calculations ($\sum_g\nu\Sigma_{f,g}\phi_g$), fission heat
source distribution ($\sum_g\epsilon\Sigma_{f,g}\phi_g$), and graphite wall temperature
distribution ($T_{wall}$) to \gls{SAM}. These 3-D quantities are collapsed into 1-D distributions
using LayeredAverage and LayeredSideAverage UserObjects from MOOSE before being transferred to
corresponding 1-D regions in \gls{SAM} using MultiAppGeneralFieldUserObjectTransfer. \gls{SAM}
solves the 1-D multichannel thermal-hydraulics system and six-group precursor equations and
transfers the fluid temperature distribution ($T_{fluid}$), wall heat transfer coefficient
distribution ($h_{wall}$), and delayed neutron source distribution ($\sum_i \lambda_i C_i$) to
Moltres. These 1-D quantities are projected to the 3-D Moltres problem domain using
MultiAppGeneralFieldNearestLocationTransfer. Consequently, the fluid temperature and delayed
neutron source are radially uniform within each fuel channel.

This work used the OpenMC Monte Carlo particle transport code to generate material-wise group
constants for Moltres with reference zero-power critical core specifications from the \gls{MSRE}
benchmark report \cite{fratoni_molten_2020}. We obtained the coupled steady-state
solution using a two-step process. First, we ran the coupled model with \gls{SAM} in time-dependent
solver mode and Moltres in $k$-eigenvalue solver mode, scaled to a fixed total power of 8 MW. Then,
we used the solution as an initial condition for a loosely coupled transient simulation with
timestep sizes starting from $dt=0.2$ s and increasing to $dt=2$
s to obtain the steady-state solutions for neutron flux, precursor, velocity, and temperature
distributions. \gls{SAM} ran in time-dependent solver mode on both instances. All simulations ran
on five compute nodes in the Lonestar6 high performance computing system at the Texas Advanced
Computing Center (TACC) with 128 MPI ranks and 512 GB memory per node. The \gls{SAM} subapp was
limited to 128 processors overall to avoid incurring performance penalties of splitting the small
1-D simulation relative to the large 3-D Moltres simulation.

\section{SIMULATION RESULTS \& DISCUSSION}

(Due to technical issues with the full-core multiphysics model, this draft presents \gls{SAM} TH
results from a multiphysics simulation of a smaller 97-channel model, and Moltres neutronics
results from a single-physics full-core simulation. We are confident that we will have
the full-core multiphysics results by the revised paper submission deadline.)

\begin{figure}[!b]
    \centering
    \includegraphics[width=0.6\columnwidth]{steady-state}
    \caption{Vertical cross section of the steady-state temperature distribution in the 3-D core
      region and the 1-D out-of-core salt loop region.}
    \label{fig:steady}
\end{figure}

This work presents the steady-state solution of the 3-D/1-D multiphysics modeling framework applied
to the \gls{MSR} described in \ref{sec:model}. Figure \ref{fig:steady} shows the steady-state
temperature distribution in a vertical cross section of the 3-D core region and the 1-D out-of-core
loop region. The striping observed in the 3-D core region illustrate the temperature differences
between the fuel and graphite regions. There is a difference of approximately
8 K between the fuel channel outlet temperature and the maximum temperature in neighboring
graphite regions. At steady state, the reactor power output and heat removal rate in the heat
exchanger component are consistent with each other and maintained a steady temperature
distribution. In the absence of control rods, we applied a fixed scaling factor to the fission
source term in Moltres to attain a core inlet temperature of approximately 908.15 K (derived from
\gls{MSRE} operating data). The neutronics model was otherwise allowed to freely respond to
temperature reactivity feedback and delayed neutron precursor drift.

Figure \ref{fig:flux} shows a vertical cross section view of the group 1 and 2 neutron fluxes. The
flux distributions exhibit expected cosine-shape distributions that peak near the core center and
drop near the external vacuum boundaries. Future publications of this model will include control
rods at the center of the core. The delayed neutron source distribution across the entire salt loop
can be seen in Figure \ref{fig:delayed}. The upward advection of delayed neutron precursors shifts
the peak significantly higher than the flux peaks. The salt recirculation time of approximately 25
s allows some precursors with relatively longer half-lives to re-enter the core.

\begin{figure}[!tb]
    \centering

    \begin{subfigure}[b]{0.45\textwidth}
      \centering
      \includegraphics[width=\textwidth]{group-1}
      \caption{Group 1}
      \label{fig:g1}
    \end{subfigure}
    \hfill
    \begin{subfigure}[b]{0.45\textwidth}
      \centering
      \includegraphics[width=\textwidth]{group-2}
      \caption{Group 2}
      \label{fig:g2}
    \end{subfigure}
    
    \caption{Group 1 and 2 neutron flux distributions on the vertical cross section of the 3-D
    core region.}
    \label{fig:flux}
\end{figure}

\begin{figure}[!htb]
    \centering
    \includegraphics[width=0.6\columnwidth]{delayed}
    \caption{Delayed neutron source distribution from the 1-D multichannel \gls{MSR}
    model.}
    \label{fig:delayed}
\end{figure}

\section{CONCLUSIONS}

This work introduced a multiscale 3-D/1-D multiphysics simulation framework for molten salt reactor
analysis. This new multiphysics tool couples Moltres, a 3-D \gls{MSR} multiphysics code, to
\gls{SAM}, a modern reactor systems code, through native \gls{MOOSE} coupling interfaces supported
by both codes. This multiscale approach balances model fidelity and computational
cost without losing 3-D spatial resolution of the reactor core.

We demonstrated its capabilities through a steady-state simulation of a full-core \gls{MSR}
model derived heavily from the \gls{MSRE} design. The model reached steady state and produced
expected distributions of various neutronic and thermal-hydraulic parameters. Further code
verification work is ongoing to certify that the Moltres, \gls{SAM}, and their coupling is
consistent with other multiphysics tools in modeling \glspl{MSR} through the \gls{MSR} progression
problem benchmarks \cite{gentry_graphite-moderated_2025}.

Ongoing and future efforts include refining the coupling scheme to improve computational
performance, adding bypass flow and other omitted components to the \gls{MSR} model, implementing
accurate control rod modeling capabilities, and performing transient \gls{MSR} analyses.

\section*{ACKNOWLEDGEMENTS}

This work was supported by the State of Texas through the Digital Molten Salt Reactor Initiative.
The authors acknowledge the Texas Advanced Computing Center (TACC) at The University of Texas at
Austin for providing computational resources that have contributed to the research results reported
within this paper.

\bibliographystyle{physor2026}
\bibliography{main}

%\appendix
%
%\section{}
%If necessary, include Appendices numbered in upper case alphabetical order.
%
%To ensure a uniform, professional look at the proceedings, please only modify the format of this template after checking with the publication chair first.


\end{document}
