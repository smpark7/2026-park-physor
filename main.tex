\documentclass[letterpaper]{physor2026}

%%% Packages Required by Class (already included)
% fancyhdr
% lastpage
% titling
% titlesec
% ragged2e
% enumitem
% amsmath
% graphicx
% geometry
% newtxtext
% newtxmath
% hyperref
% cleveref
% caption
% authblk
% apptools
% appendix
% ifpdf
% epstopdf

%%% Some other useful packages
% \usepackage{tikz}
% \usepackage{color}
\usepackage{subcaption}
% \usepackage{algcompatible}
% \usepackage{bm}
% \usepackage{array}

\usepackage{float}
\newfloat{textbox}{htbp}{lob}

\usepackage{siunitx}
\graphicspath{{images/}}
\usepackage[acronym]{glossaries}
\usepackage{nomencl} % If needed
\makenomenclature


\title{A Multiscale 3-D/1-D Multiphysics Molten Salt Reactor Simulation Framework Based on
Moltres-SAM Coupling}

%%% Authors (use arabic numbers: 1, 2, 3, etc. for affiliationNumber)
%%% \addAuthor{GivenName MiddleInitial. FamilyName}{affiliationNumber}
\addAuthor[sunmyung.park@austin.utexas.edu]{Sun Myung Park}{1}
\addAuthor{Cole A.\ Gentry}{1}
\addAuthor{Nicholas Luciano}{1}
\addAuthor{Kevin T.\ Clarno}{1}
\addAuthor{Rok Krpan}{2}
\addAuthor{Carlo Fiorina}{2}
\addAuthor{Jean Ragusa}{2}

%%% Affiliations (from authblk)
%%% \addAffiliation{affiliationNumber}{Name of Institute, City, State/Country}
\addAffiliation{1}{The University of Texas at Austin, Austin, TX, USA}
\addAffiliation{2}{Texas A\&M University, College Station, TX, USA}

%%% Write text for abstract
%%% Most text modifying commands will work in abstract
\Abstract{%
Reactor analysis tools for molten salt reactors (MSRs) are critical for supporting their
development and commercial deployment. However, the liquid fuel form in MSRs poses computational
challenges such as strong temperature reactivity feedback and delayed neutron precursor drift.
This work introduces a multiscale 3-D/1-D multiphysics simulation framework for molten salt reactor
analysis by coupling two MOOSE-based codes, Moltres and SAM, for neutronics and thermal-hydraulics,
respectively. This new multiphysics tool aims to provide a means for engineering-level MSR analyses
while retaining 3-D spatial resolution of the core for accurate reactivity feedback and precursor
tracking. We present the coupling scheme between Moltres and SAM implemented entirely with native
coupling interfaces provided by MOOSE. Thereafter, we demonstrate the multiphysics tool on a
full-core MSR model derived from the Molten Salt Reactor Experiment (MSRE). The steady-state
neutronics and thermal-hydraulics results were consistent with expected behavior of
a graphite-moderated MSR. Ongoing work focuses on improving model fidelity and computational
performance and verifying results against other multiphysics tools.
}


%%% List up to 5 keywords separated by a comma
\keywords{molten salt reactor, multiphysics, multiscale, MOOSE}

%%% Provide a short title for the header on odd pages
\shortTitle{A Multiscale 3-D/1-D Multiphysics Molten Salt Reactor Simulation Framework}

%%% Provide a short author listing for the header on even pages
\authorHead{S. P, et al.}

%%% If LaTeX reports the line number of an error at \begin{document} it
%%%   is most likely due to an error in one of the commands above
\begin{document}
\include{acros}

\section{INTRODUCTION}\label{sec:introduction}

\glspl{MSR} have gained renewed interest as a promising Generation IV reactor concept due to their
strong negative reactivity feedbacks, high operating temperature, and potential for online fuel
reprocessing. Unlike solid-fuel reactors, \glspl{MSR} feature mobile fuel inventory, and the
\glspl{DNP} drift with the circulating fuel salt. Therefore, the reactor kinetics
are tightly coupled with thermal-hydraulics.

Developing reduced-order models for \glspl{MSR}, whether for digital twins or engineering-level
tasks, will require both experimental data and high-fidelity simulation data. Two high-fidelity
tools, VERA \cite{graham_multiphysics_2021} and GeN-Foam \cite{fiorina_gen-foam_2022},
are being developed at The University of Austin and Texas A\&M University for \gls{MSR} analysis
\cite{gentry_graphite-moderated_2025} under the Digital Molten Salt Reactor Initiative. In this
context, our requirements for ``high-fidelity'' \gls{MSR} tool are: modeling the full salt loop,
modeling every channel discretely, accounting for 3-D heat conduction, tracking \gls{DNP} movement,
and modeling steady state and various transient scenarios.

%Simple 1-D flow models that model the reactor as a single or a few channels can be well-calibrated
%to reproduce integral reactor responses (e.g., $k_{\text{eff}}$, $\beta$), but may not be reliable
%for transient scenarios with significant spatial shifts in temperature and neutron flux.
%On the other hand, high-resolution 3-D computational fluid dynamics models are valuable for
%generating reference solutions and closure relations for reduced-order models but remain
%computationally expensive \cite{merzari_large-scale_2017}.

In this work, we introduce a new high-fidelity, multiscale 3-D/1-D multiphysics \gls{MSR} analysis
tool built by coupling two \gls{MOOSE}-based applications, Moltres
\cite{lindsay_introduction_2018} (3-D \gls{MSR} multiphysics code) and \gls{SAM} \cite{hu_sam_2025}
(modern nuclear system code for advanced reactor safety analysis). This research effort aims to
leverage advanced simulation capabilities borne from the \gls{NEAMS} program to build a
\gls{MOOSE}-based tool that can provide high-fidelity 3-D solutions at resolutions similar to VERA
for comparative analysis. This work also builds on previous development work on Moltres for
\gls{MSR} analysis \cite{park_advancement_2020, park_verification_2022, park_advancements_2025}.

\section{SOFTWARE DESCRIPTION}

\subsection{MOLTRES}

Moltres is an open-source multiphysics reactor simulation software developed for \gls{MSR}
modeling. Moltres solves the neutrondiffusion or $S_N$ neutron transport equations for neutronics
modeling. The \gls{DNP} model in Moltres includes advection capabilities that can natively couple
to any flow model in \gls{MOOSE} such as the Navier-Stokes module or the \gls{SAM} nuclear system
code. These modeling capabilities have been verified in code-to-code verification studies for
\gls{MSR}-specific phenomena
\cite{park_advancement_2020, park_verification_2022, park_advancements_2025}.

Moltres has been used in past work to model the \gls{MSRE}
\cite{lindsay_introduction_2018, park_advancements_2025}, \gls{MSFR} \cite{park_advancement_2020},
and fast-spectrum chloride \glspl{MSR} \cite{lawson_development_2024}. A key factor for selecting
Moltres for this research is its integration within the \gls{MOOSE}
ecosystem. As a \gls{MOOSE}-based tool, Moltres benefits from highly scalable mesh and
solver routines available through \gls{MOOSE} and natively couples to \gls{SAM} (also a
\gls{MOOSE}-based tool).
%Moltres' open-source nature accelerates the development and testing of new
%features to improve \gls{MSR} modeling and simulation performance.

\subsection{SYSTEM ANALYSIS MODULE (SAM)}

\gls{SAM} \cite{hu_sam_2025} is a modern system analysis tool for advanced reactor design
optimization, safety
analyses, and licensing support. Though originally developed for sodium fast reactors, \gls{SAM}
developers have implemented and tested numerous advanced features tailored for salt-cooled reactors
such as multiphase species transport, freeze plug modeling, and drift flux modeling. In most
applications, \gls{SAM} solves the 1-D incompressible but thermally expandable flow equations along
with heat transfer equations. \gls{SAM} also supports coarse-mesh, multi-D flow modeling typically
used for reactor designs with large coolant pools or porous medium flow in pebble-bed reactors.
The present work uses \gls{SAM}-Lite, a non-export controlled version of \gls{SAM} with some
restricted capabilities removed. Since all capabilities demonstrated in this work can be replicated
using \gls{SAM}, we drop the \textit{-Lite} suffix for brevity.

In general, two different \gls{SAM} \gls{MSRE} models have been demonstrated in the current
literature. One \gls{SAM} model approximates the \gls{MSRE} core as a single 1-D pipe
while preserving the height and flow area \cite{hu_fy21_2021}. This model has been validated
against experimental data
from various transient experiments by accurately reproducing integral neutronic and temperature
metrics. The other \gls{SAM} model couples to the Griffin reactor physics code and
models the \gls{MSRE} core as a homogenized, 2-D axisymmetric region
\cite{jaradat_multiphysics_2023}. Griffin solves the neutron diffusion
equations and \gls{SAM} solves porous media flow equations in the 2-D core region. In both
\gls{SAM}-based models, \gls{SAM} models the ex-core salt loop as several serially connected 1-D
pipe regions forming a loop with the core region.

\subsection{MOLTRES-SAM COUPLING}

\begin{figure}[!htb]
    \centering
    \includegraphics[width=0.8\columnwidth]{coupling}
    \caption{Moltres and \gls{SAM} MultiApp coupling setup and data transfers.}
    \label{fig:coupling}
\end{figure}

Moltres-SAM multiphysics simulations couple Moltres and \gls{SAM} as \textit{sibling subapps} using
the MOOSE MultiApp system. A dummy \textit{main app} dictates all data transfers between
the two subapps but does not perform any calculations. Figure \ref{fig:coupling} shows an overview
of how Moltres and \gls{SAM} are
coupled and the data transfers involved between the two codes.

Moltres solves the 3-D two-group neutron diffusion and graphite heat conduction equations and
transfers neutron source distribution for precursor source calculations
($\sum_g\nu\Sigma_{f,g}\phi_g$), fission heat source distribution
($\sum_g\epsilon\Sigma_{f,g}\phi_g$), and graphite wall temperature
distribution ($T_{wall}$) to \gls{SAM}. These 3-D quantities are collapsed into 1-D distributions
using LayeredAverage and LayeredSideAverage UserObjects from MOOSE before being transferred to
corresponding 1-D regions in \gls{SAM}. Moltres uses material-wise group constants generated using the
OpenMC Monte Carlo particle transport calculations on a reference zero-power critical core. 

\gls{SAM} solves the 1-D multichannel thermal-hydraulics system and six-group precursor equations
and transfers the fluid temperature distribution ($T_{fluid}$), wall heat transfer coefficient
distribution ($h_{wall}$), and delayed neutron source distribution ($\sum_i \lambda_i C_i$) to
Moltres. The data transfer process projects these distributions from each 1-D salt channel to the
3-D Moltres model. Consequently, the fluid temperature and delayed neutron source in the
3-D model are radially uniform within each fuel channel.

\section{MODEL DESCRIPTION} \label{sec:model}

%Figure \ref{fig:msre} shows a vertical
%cross section view of the \gls{MSRE} geometry.

%\begin{figure}[h]
%    \centering
%    \includegraphics[width=0.5\columnwidth]{msre-picture}
%    \caption{Vertical cross section of the \gls{MSRE} vessel \cite{robertson_msre_1965}.}
%    \label{fig:msre}
%\end{figure}

\subsection{REACTOR SPECIFICATIONS}

The \gls{MSR} model is a preliminary design being developed for the \gls{MSR}
progression problems \cite{gentry_graphite-moderated_2025}, a series of openly available \gls{MSR}
benchmark problems to facilitate code-to-code verification between the Moltres-SAM tool, VERA,
GeN-Foam, and other \gls{MSR} tools. This \gls{MSR} is based on the \gls{MSRE} built in 1964 at Oak
Ridge National Laboratory \cite{robertson_msre_1965}. The model incorporates most of the \gls{MSRE}
design as a representative model of graphite-moderated \glspl{MSR}. Several geometric
approximations have been made to simplify model creation for code-to-code verification between
multiphysics \gls{MSR} analysis tools of varying model fidelity. The primary goals of testing the
coupled Moltres-\gls{SAM} tool in this study are to evaluate the feasibility of
Moltres-\gls{SAM} coupling, assess its current capabilities, and determine the computing resources
needed to model a full-core problem.

\begin{figure}[!tb]
    \centering
    \begin{subfigure}[b]{0.375\textwidth}
      \centering
      \includegraphics[width=\textwidth]{3d-geometry}
      \caption{3-D Moltres model with a cutout}
      \label{fig:3d}
    \end{subfigure}
    \hfill
    \begin{subfigure}[b]{0.615\textwidth}
      \centering
      \includegraphics[width=\textwidth]{1d-geometry}
      \caption{1-D multichannel \gls{SAM} model}
      \label{fig:1d}
    \end{subfigure}
    \caption{3-D Moltres and 1-D multichannel \gls{SAM} \gls{MSR} model geometries. The cutout
    in the 3-D model illustrates the internal fuel channels.}
    \label{fig:geometry}
\end{figure}

\begin{table}[!bt]
  \small
  \centering
  \caption{Key specifications of the \gls{MSR} core model.}
  \label{table:dimensions}
  \begin{tabular}{l S}
    \toprule
    Parameter & {Value} \\
    \midrule
    Quarter-core power [MW] & 2.0 \\
    Lower plenum height [m] & 0.1875 \\
    Graphite height [m] & 1.70027 \\
    Upper plenum height [m] & 0.254 \\
    Core radius [m] & 0.7366 \\
    Central channel inner radius [m] & 0.0254 \\
    Central channel outer radius [m] & 0.031992 \\
    Total mass flow rate [kg s$^{-1}$] & 37.289 \\
    \bottomrule
  \end{tabular}
\end{table}

Figure \ref{fig:3d} shows the \gls{MSR} quarter-core geometry. It consists of 284 regular fuel
channels and two central, annular salt half-channels. The fuel channel centroids form a
0.0508~m$\times$0.0508~m square grid in the graphite matrix. These annular channels surround
control rod thimbles which were omitted in this work due to incompatibility of the control rod and
gas regions with the present neutron diffusion method. We plan to incorporate these transport
effects in future studies. The present study also simplifies the peripheral core region by
replacing the \gls{MSRE} annular bypass channel, core can structure, and downcomer regions with
graphite.
The fuel-graphite core region is sandwiched by the upper and lower plena regions filled with 100~\%
fuel salt and surrounded radially by the same vessel wall. Table \ref{table:dimensions} lists
key specifications of the \gls{MSR} model.

Salt flows upwards through the core and flows through a pump and a heat exchanger before
re-entering the core through a downcomer. The present work assumes a uniform gamma heat source in
the graphite regions cumulatively equal to 7.8~\% of total fission heat produced.

\subsection{SIMULATION SETUP}

%Moltres provides 3-D neutron diffusion,
%precursor source, and graphite heat conduction modeling while \gls{SAM} provides 1-D multichannel
%flow modeling for system thermal-hydraulics and precursor drift to model all salt channels in a
%full-core \gls{MSR} geometry and ex-core segments of the primary loop. The 1-D multichannel flow
%model reduces the computational cost relative to 3-D flow models while simultaneously retaining
%more 3-D spatial resolution relative to single-/few-channel 1-D flow models. The long-term
%objective of this research is to develop an intermediate-fidelity \gls{MSR} simulation framework to
%support the digital twin development efforts under the Digital Molten Salt Reactor Initiative at
%The University of Texas at Austin.

Moltres models the 3-D core domain, while \gls{SAM} models the primary salt loop as a closed 1-D
flow loop, with 1-D pipes corresponding to each fuel channel and plena regions, as shown in Figures
\ref{fig:3d} and \ref{fig:1d}, respectively. We did not apply form loss coefficients on any of the
fuel channel entrances in the present study. Out-of-core
\gls{SAM} components were adapted from an existing \gls{SAM} \gls{MSRE} model \cite{hu_fy21_2021}
openly available in the Virtual Test Bed Repository \cite{giudicelli_virtual_2023}. The 3-D domain
consists of $\sim$200,000 2nd-order mesh elements. The 1-D domain consists of $\sim$3000
2nd-order mesh elements. Both meshes consist of 15 axial mesh segments in the core region.

We ran a pseudo-transient simulation on 2-second timesteps towards steady state with explicit
coupling between Moltres (steady-state) and SAM (transient).
The $k$-eigenvalue neutronics calculation and steady-state heat
conduction calculation in Moltres are fully coupled in a single \gls{PJFNK} solve. We fixed total
quater-core power to 2~MW. The \gls{SAM} heat exchanger component contains a heat removal term of
the form $[h(T-824.8167)]$, where $h$ is the heat transfer coefficient and $T$ is the salt
temperature, and is tuned to obtain a core inlet temperature of 908.15~K at steady state.

The simulation ran on one CPU node consisting of two AMD EPYC 9754 128-core processors and took
$\sim$1 minute of wall time per time step, of which data transfers comprised roughly
half. We limited the \gls{SAM} subapp to 16 MPI ranks to avoid excessive parallelization and limit
its memory usage from replicated mesh storage. The Moltres subapp runs on a distributed memory mesh
across all 256 MPI ranks. Current investigations are ongoing to run Moltres-\gls{SAM} simulations
on more than one compute node. By extrapolation, we expect a full-core, iteratively coupled
simulation to take 3 wall minutes per time step on 8 128-core processors, with additional
computational headroom for strong scaling by adding more processors.

%We obtained the coupled steady-state
%solution using a two-step process. First, we ran the coupled model with \gls{SAM} in time-dependent
%solver mode and Moltres in $k$-eigenvalue solver mode, scaled to a fixed total power of 8 MW. Then,
%we used the solution as an initial condition for a loosely coupled transient simulation with
%timestep sizes starting from $dt=0.2$~s and increasing to $dt=2$~s
%to obtain the steady-state solutions for neutron flux, precursor, velocity, and temperature
%distributions. \gls{SAM} ran in time-dependent solver mode on both instances. All simulations ran
%on five compute nodes in the Lonestar6 high performance computing system at the Texas Advanced
%Computing Center (TACC) with 128 MPI ranks and 512~GB memory per node. The \gls{SAM} subapp was
%limited to 128 processors overall to avoid incurring performance penalties of splitting the small
%1-D simulation relative to the large 3-D Moltres simulation. Each timestep took approximately two
%minutes, out of which 40 s were spent on data transfers.

\section{SIMULATION RESULTS \& DISCUSSION}

\begin{table}[!b]
  \centering
  \small
  \caption{Steady-state reactor results}
  \begin{tabular}{l S}
    \toprule
    Parameter & {Value} \\
    \midrule
    Inlet temperature [K] & 908.15 \\
    Outlet temperature [K] & 930.63 \\
    Salt circulation time [s] & 25.1 \\
    Delayed neutron fraction [-] & 0.00526 \\
    \bottomrule
  \end{tabular}
  \label{table:results}
\end{table}

Figure \ref{fig:temperature} shows the steady-state temperature distributions in the 3-D graphite
region and the 1-D fuel salt loop. Both graphite and fuel salt temperatures peak at the top
central region of the core. The two annular half-channels report the hottest fuel salt temperature
due to their central location and slower flow velocities. This would change in future studies when
form loss coefficients are applied to replicate the \gls{MSRE} radial flow profile. Excluding the
annular channels, the difference between peak graphite and fuel temperatures is $\sim$13~K, close
to $\sim$19~K estimated in the \gls{MSRE} \cite{robertson_msre_1965}. We expect to obtain better
agreement in future studies once the flow profile and graphite heating distribution match
\gls{MSRE} specifications. The
temperature difference between the inlet and outlet temperatures in Table \ref{table:results}
agrees with energy balance calculations for 2~MW power within 0.007~\%. The salt circulation time
of 25.1~s also agrees closely with \gls{MSRE} salt circulation time.

The observed delayed neutron fraction $\beta$ is 0.00526, corresponding to a $\beta$ loss of
0.00126. The loss value is lower than the $\beta_\text{eff}$ value of 0.00225 reported in the
\gls{MSRE} benchmark report \cite{fratoni_molten_2020}. We attribute the discrepancy to geometrical
differences in the model and the unweighted nature of our $\beta$ value. The model in this work
assumes that the lower plenum is 100~\% fuel salt but the \gls{MSRE} has support grid structures
accounting for 9.2~\% of the lower plenum volume. The \gls{MSRE} exhibits faster flow rates in the
center of the core \cite{robertson_msre_1965} which also contributes to pushing more precursors out
of the core before they decay. Nevertheless, the axial \gls{DNP} production and
decay distributions in Figure \ref{fig:precursor} agrees qualitatively with Figure
\ref{fig:benchmark} from the \gls{MSRE} benchmark report.

\begin{figure}[!tb]
    \centering
    \includegraphics[width=0.8\columnwidth]{temperature-label}
    \caption{Graphite and fuel salt temperature distributions at steady-state.}
    \label{fig:temperature}
\end{figure}

\begin{figure}[!htb]
    \centering
    \begin{subfigure}[b]{0.45\textwidth}
      \centering
      \includegraphics[width=\textwidth]{precursor}
      \caption{Axial \gls{DNP} production and decay distributions from \gls{SAM}.}
      \label{fig:precursor}
    \end{subfigure}
    \hfill
    \begin{subfigure}[b]{0.54\textwidth}
      \centering
      \includegraphics[width=\textwidth]{benchmark}
      \caption{Probabilities of \gls{DNP} decay in the stationary (orange) and
        circulating (blue) fuel cases from the \gls{MSRE} benchmark.}
      \label{fig:benchmark}
    \end{subfigure}
    \caption{Comparison of \gls{DNP} distributions in this study to the \gls{MSRE}
      benchmark report \cite{fratoni_molten_2020}. Discontinuities occur at graphite-plena
      interfaces due to changes in the flow area.}
    \label{fig:precursors}
\end{figure}

%Figure \ref{fig:flux} shows a vertical cross section view of the group 1 and 2 neutron fluxes. The
%flux distributions exhibit expected cosine-shape distributions that peak near the core center and
%drop near the external vacuum boundaries. The flux peak magnitudes agree well qualitatively with
%results of a similar 2-D axisymmetric \gls{MSRE} model by Reynolds \& Palmer
%\cite{reynolds_analysis_2023} as shown in Figure \ref{fig:quasimolto}. We plan to include control
%rods at the center of the core in future publications.
%The delayed neutron source distribution across the entire salt loop
%can be seen in Figure \ref{fig:delayed}. The upward advection of \glspl{DNP} shifts
%the peak towards the top of the core. The salt recirculation time of approximately 25~s
%allows some precursors with relatively longer half-lives to re-enter the core.
%
%\begin{figure}[!tb]
%    \centering
%
%    \begin{subfigure}[b]{0.4\textwidth}
%      \centering
%      \includegraphics[width=\textwidth]{group-1}
%      \caption{Group 1}
%      \label{fig:g1}
%    \end{subfigure}
%    \hfill
%    \begin{subfigure}[b]{0.4\textwidth}
%      \centering
%      \includegraphics[width=\textwidth]{group-2}
%      \caption{Group 2}
%      \label{fig:g2}
%    \end{subfigure}
%    
%    \caption{Group 1 and 2 neutron flux distributions on the vertical cross section of the 3-D
%    core region.}
%    \label{fig:flux}
%
%    \includegraphics[width=0.5\columnwidth]{quasimolto}
%    \caption{Group 1 and 2 neutron flux distributions from a 2-D axisymmetric \gls{MSRE} study by
%    Reynolds \& Palmer \cite{reynolds_analysis_2023} for comparison of flux shape and magnitude.}
%    \label{fig:quasimolto}
%\end{figure}

\section{CONCLUSIONS}

This work introduced a multiscale 3-D/1-D multiphysics simulation framework for molten salt reactor
analysis. This new multiphysics tool couples Moltres, a 3-D \gls{MSR} multiphysics code, to
\gls{SAM}, a modern reactor systems code, through native \gls{MOOSE} coupling interfaces supported
by both codes. This multiscale approach balances model fidelity and computational
cost without losing 3-D spatial resolution of the reactor core.

We demonstrated its capabilities through a steady-state simulation of a full-core \gls{MSR}
model derived heavily from the \gls{MSRE} design. The model reached steady state and produced
expected distributions of various neutronic and thermal-hydraulic parameters. Further code
verification work is ongoing to certify that the Moltres, \gls{SAM}, and their coupling are
consistent with other multiphysics tools in modeling \glspl{MSR} through the \gls{MSR} progression
problem benchmarks \cite{gentry_graphite-moderated_2025}.

Ongoing and future efforts include refining the coupling scheme to improve computational
performance, adding bypass flow and other omitted components to the \gls{MSR} model, implementing
accurate control rod modeling capabilities, and performing transient \gls{MSR} analyses.

\section*{ACKNOWLEDGEMENTS}
{%\small
This work was supported by the State of Texas through the Digital Molten Salt Reactor Initiative.
The authors acknowledge the Texas Advanced Computing Center (TACC) at The University of Texas at
Austin for providing computational resources that have contributed to the research results reported
within this paper. The authors thank the SAM development team at Argonne National Laboratory
for their technical support.
}

{%\small
\bibliographystyle{physor2026}
\bibliography{main}
}

%\appendix
%
%\section{}
%If necessary, include Appendices numbered in upper case alphabetical order.
%
%To ensure a uniform, professional look at the proceedings, please only modify the format of this template after checking with the publication chair first.


\end{document}
