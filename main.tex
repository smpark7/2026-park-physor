\documentclass[letterpaper]{physor2026}

%%% Packages Required by Class (already included)
% fancyhdr
% lastpage
% titling
% titlesec
% ragged2e
% enumitem
% amsmath
% graphicx
% geometry
% newtxtext
% newtxmath
% hyperref
% cleveref
% caption
% authblk
% apptools
% appendix
% ifpdf
% epstopdf

%%% Some other useful packages
% \usepackage{tikz}
% \usepackage{color}
\usepackage{subcaption}
% \usepackage{algcompatible}
% \usepackage{bm}
% \usepackage{array}

\usepackage{float}
\newfloat{textbox}{htbp}{lob}

\usepackage{placeins}
\usepackage{siunitx}
\graphicspath{{images/}}
\usepackage[acronym]{glossaries}
\usepackage{nomencl} % If needed
\makenomenclature
\include{acros}


\title{A Multiscale 3-D/1-D Multiphysics Molten Salt Reactor Simulation Framework Based on
Moltres-SAM Coupling}

%%% Authors (use arabic numbers: 1, 2, 3, etc. for affiliationNumber)
%%% \addAuthor{GivenName MiddleInitial. FamilyName}{affiliationNumber}
\addAuthor[sunmyung.park@austin.utexas.edu]{Sun Myung Park}{1}
\addAuthor{Cole A.\ Gentry}{1}
\addAuthor{Nicholas Luciano}{1}
\addAuthor{Kevin T.\ Clarno}{1}
\addAuthor{Rok Krpan}{2}
\addAuthor{Carlo Fiorina}{2}
\addAuthor{Jean Ragusa}{2}

%%% Affiliations (from authblk)
%%% \addAffiliation{affiliationNumber}{Name of Institute, City, State/Country}
\addAffiliation{1}{The University of Texas at Austin, Austin, TX, USA}
\addAffiliation{2}{Texas A\&M University, College Station, TX, USA}

%%% Write text for abstract
%%% Most text modifying commands will work in abstract
\Abstract{%
\gls{MSR} analysis tools are critical for supporting \gls{MSR} development and licensing.
However, the circulating liquid fuel introduces unique computational challenges, including strong
temperature reactivity feedback and the advection of delayed neutron precursor. This work
introduces a coupled multiscale 3-D/1-D multiphysics framework for channel-type \glspl{MSR}
by linking two MOOSE-based codes: Moltres (neutronics) and SAM (thermal-hydraulics). This approach
leverages computationally efficient system-level thermal-hydraulics while retaining the 3-D spatial
resolution required for accurate precursor tracking and reactivity feedback in the core. We present
the development and initial verification of this coupling using a quarter-core graphite-moderated
\gls{MSR} model. Steady-state results demonstrate that the coupled framework shows good agreement
with analytical solutions and reference data. The developed framework provides a physics-based
foundation to support ongoing data-driven digital twin development and enables comparative analyses
against other multiphysics tools with differing model resolutions, such as the VERA simulation code
suite and GeN-Foam multiphysics solver for reactor analysis.
}

%%% List up to 5 keywords separated by a comma
\keywords{molten salt reactor, multiphysics, multiscale, MOOSE, MSR}

%%% Provide a short title for the header on odd pages
\shortTitle{A Multiscale 3-D/1-D Multiphysics Molten Salt Reactor Simulation Framework}

%%% Provide a short author listing for the header on even pages
\authorHead{S. P, et al.}

%%% If LaTeX reports the line number of an error at \begin{document} it
%%%   is most likely due to an error in one of the commands above
\begin{document}

\section{INTRODUCTION}\label{sec:introduction}

\glspl{MSR} have gained renewed interest as a promising Generation IV reactor concept due to their
strongly negative reactivity feedback, high operating temperature, and potential for online fuel
reprocessing. Unlike solid-fuel reactors, \glspl{MSR} feature mobile fuel inventory, and the
\glspl{DNP} drift with the circulating fuel salt. Therefore, the reactor kinetics
are tightly coupled with thermal-hydraulics.

Developing reduced-order models for \glspl{MSR} for digital twins or engineering-level
tasks will require both experimental data and reliable simulation data. In collaborative research
efforts at The University of Texas at Austin and Texas A\&M University, two multiphysics
tools, VERA \cite{graham_multiphysics_2021} and GeN-Foam \cite{fiorina_gen-foam_2022}, have been
adopted for \gls{MSR} analysis and digital twin development
under the Digital Molten Salt Reactor Initiative. These reactor analysis tools must demonstrate
robustness and accuracy over a wide range of \gls{MSR} simulation scenarios to produce reliable
training and reference data for reduced-order models. To that end, a code-to-code verification
exercise is underway through a series of \gls{MSR} progression problems
\cite{gentry_graphite-moderated_2025} based on the \gls{MSRE} reactor design.

In this work, we introduce a new multiscale 3-D/1-D multiphysics \gls{MSR} analysis tool built by
coupling two \gls{MOOSE} \cite{harbour_40_2025} applications: Moltres
\cite{lindsay_introduction_2018} and \gls{SAM} \cite{hu_sam_2025}. While Moltres itself is
a multiphysics solver for \glspl{MSR} with access to flow physics modules from the \gls{MOOSE}
framework, the Moltres-SAM coupling leverages on an efficient nuclear system code developed under
the \gls{NEAMS} program with specific features tailored for advanced reactor safety analysis. This
work also benefits from recent development work on Moltres for \gls{MSR} analysis
\cite{park_advancement_2020, park_verification_2022, park_advancements_2025}.

The 3-D/1-D approach couples a 3-D neutronics and solid heat conduction model in Moltres to a
1-D multichannel model in \gls{SAM}. The multichannel model consists of a 1-to-1 ratio of 1-D pipes
to the fuel channels in an \gls{MSR} and the out-of-core components to form a closed salt
loop including a pump and a heat exchanger. This contrasts with existing \gls{SAM} models that
either model the core as a single pipe \cite{hu_fy21_2021} or a homogenized 2-D axisymmetric porous
media region \cite{jaradat_multiphysics_2023}. With
almost 300 1-D components for a quarter-core \gls{MSR} model, this work presents one of the largest
channel-type models ever run in \gls{SAM}. The primary goals of this study are to demonstrate the
viability of the large multichannel models in \gls{SAM} and perform initial verification of the
Moltres-\gls{SAM} coupling through pseudo-transient simulations for steady-state results of a
quarter-core graphite-moderated \gls{MSR}.

\section{SOFTWARE DESCRIPTION}

\subsection{MOLTRES}

Moltres is an open-source multiphysics reactor simulation software developed for \gls{MSR}
modeling. Moltres solves the neutron diffusion or $S_N$ neutron transport equations for neutronics
modeling. The \gls{DNP} model in Moltres supports coupling with any flow model in the \gls{MOOSE}
Navier-Stokes physics module. These modeling capabilities have been verified in code-to-code
verification studies for \gls{MSR}-specific phenomena
\cite{park_advancement_2020, park_verification_2022, park_advancements_2025}.

Moltres has been used in past work to model the \gls{MSRE}
\cite{lindsay_introduction_2018, park_advancements_2025}, \gls{MSFR} \cite{park_advancement_2020},
and fast-spectrum chloride \glspl{MSR} \cite{lawson_development_2024}. A key factor for selecting
Moltres for this work is its integration within the \gls{MOOSE}
ecosystem. As a \gls{MOOSE}-based tool, Moltres benefits from highly scalable mesh and
solver routines through \gls{MOOSE} and natively couples to \gls{SAM}.

\subsection{SYSTEM ANALYSIS MODULE (SAM)}

\gls{SAM} \cite{hu_sam_2025} is a modern system analysis tool for advanced reactor design
optimization, safety analyses, and licensing support. Though originally developed for sodium fast
reactors, \gls{SAM} developers have implemented and tested numerous advanced features applicable to
salt-cooled reactor analysis
such as multiphase species transport, freeze plug modeling, and drift flux modeling. In most
applications, \gls{SAM} solves the 1-D weakly compressible flow equations along
with heat transfer equations.
The present work uses \gls{SAM}-Lite, a non-export controlled version of \gls{SAM} with some
restricted capabilities removed. Since all capabilities demonstrated in this work can be replicated
using \gls{SAM}, we drop the \textit{-Lite} suffix for brevity.

In general, two different \gls{SAM} \gls{MSRE} models have been demonstrated in the current
literature. One \gls{SAM} model approximated the \gls{MSRE} core as a single 1-D pipe
while preserving the height and flow area \cite{hu_fy21_2021}. This model was validated
against experimental data
from various transient experiments by accurately reproducing integral neutronic and temperature
metrics. The other \gls{SAM} model coupled to the Griffin reactor physics code and
modeled the \gls{MSRE} core as a homogenized, 2-D axisymmetric region
\cite{jaradat_multiphysics_2023}. Griffin solved the neutron diffusion
equations and \gls{SAM} solved porous media flow equations in the 2-D core region. In both
\gls{SAM}-based models, \gls{SAM} modeled the ex-core salt loop as several serially connected 1-D
pipe regions forming a loop with the core region.

\subsection{MOLTRES-SAM COUPLING AND DATA TRANSFERS}

\begin{figure}[!htb]
    \centering
    \includegraphics[width=0.7\columnwidth]{coupling}
    \caption{Moltres and \gls{SAM} MultiApp coupling setup and data transfers.}
    \label{fig:coupling}
\end{figure}

Moltres-SAM multiphysics simulations couple Moltres and \gls{SAM} as \textit{sibling subapps} using
the MOOSE MultiApp system. A \textit{main app} dictates all data transfers between
the two subapps but does not perform any calculations. Figure \ref{fig:coupling} shows an overview
of how Moltres and \gls{SAM} are coupled and the data transfers involved between the two codes.

Moltres solves the 3-D two-group neutron diffusion and graphite heat conduction equations and
transfers neutron source distribution for precursor source calculations
($\sum_g\nu\Sigma_{f,g}\phi_g$), fission heat source distribution
($\sum_g\epsilon\Sigma_{f,g}\phi_g$), and graphite wall temperature distribution ($T_{wall}$) to
\gls{SAM}. These 3-D quantities are sampled and collapsed into 1-D distributions using
NearestPointLayeredAverage and NearestPointLayeredSideAverage UserObjects before being transferred
to corresponding 1-D regions in \gls{SAM}.

\gls{SAM} solves the 1-D multichannel thermal-hydraulics system and six-group precursor equations
and transfers the fluid temperature distribution ($T_{fluid}$), wall heat transfer coefficient
distribution ($h_{wall}$), and delayed neutron source distribution ($\sum_i \lambda_i C_i$) to
Moltres. The data transfer process projects these distributions from each 1-D salt channel to the
3-D Moltres model. Consequently, the fluid temperature and delayed neutron source in the
3-D model are radially uniform within each fuel channel.

\section{MODEL DESCRIPTION} \label{sec:model}

\subsection{MOLTEN SALT REACTOR MODEL SPECIFICATIONS}

The \gls{MSR} model is a preliminary design being developed for the \gls{MSR} progression problems
\cite{gentry_graphite-moderated_2025}, a series of \gls{MSR} benchmark problems to facilitate
code-to-code verification between the Moltres-SAM tool, VERA \cite{graham_multiphysics_2021},
GeN-Foam, and other \gls{MSR} software. The model incorporates most of the \gls{MSRE}
\cite{robertson_msre_1965} design as a representative model of graphite-moderated \glspl{MSR}.
Several geometric approximations have been made to simplify model creation for code-to-code
verification between multiphysics \gls{MSR} analysis tools of varying model fidelity.

\begin{table}[!h]
  \centering
  \begin{minipage}{0.44\textwidth}
    \footnotesize
    \centering
    \caption{Key specifications of the \gls{MSR} core model.}
    \label{table:dimensions}
    \begin{tabular}{l S}
      \toprule
      Parameter & {Value} \\
      \midrule
      Quarter-core power [MW] & 2.0 \\
      Lower plenum height [m] & 0.1875 \\
      Graphite height [m] & 1.70027 \\
      Upper plenum height [m] & 0.254 \\
      Core radius [m] & 0.7366 \\
      Annular channel inner radius [m] & 0.0254 \\
      Annular channel outer radius [m] & 0.031992 \\
      Total mass flow rate [kg s$^{-1}$] & 37.289 \\
      \bottomrule
    \end{tabular}
  \end{minipage}
  \hfill
  \begin{minipage}{0.54\textwidth}
    \footnotesize
    \centering
    \caption{Salt and graphite material properties. $T$ is temperature.}
    \label{table:properties}
    \begin{tabular}{l c}
      \toprule
      Property & {Expression} \\
      \midrule
      Salt density [kg m$^{-3}$] & {$2413 - \left(0.488\times T\right)$} \\
      Salt viscosity [Pa s] & {$1.16\times10^{-4} \times e^{(3755 / T)}$} \\
      Salt heat capacity [J kg$^{-1}$ K$^{-1}$] & 2386 \\
      Graphite density [kg m$^{-3}$] & 1860 \\
      Graphite conductivity [W m$^{-1}$ K$^{-1}$] & 154.797 \\
      Graphite heat capacity [J kg$^{-1}$ K$^{-1}$] & 1758.46 \\
      \bottomrule
    \end{tabular}
  \end{minipage}
\end{table}

Figure \ref{fig:3d} shows the \gls{MSR} quarter-core geometry. It consists of 284 regular fuel
channels and two central, annular salt half-channels. The fuel channel centroids form a
0.0508~m$\times$0.0508~m square grid in the graphite matrix. These annular channels surround
control rod thimbles which were omitted in this work due to incompatibility of the control rod and
gas regions with the present neutron diffusion method. We plan to incorporate control rods in
future studies. The present study also simplifies the peripheral core region by
replacing the \gls{MSRE} annular bypass channel, core can structure, and downcomer regions with
graphite. We do not expect these approximations to have a significant impact on
integral quantities, such as core inlet/outlet temperatures and the delayed neutron fraction, or
point quantities far from the peripheral regions, such as maximum salt and graphite temperatures.
The fuel-graphite core region is sandwiched by the upper and lower plena regions filled with 100~\%
fuel salt and surrounded radially by a vessel wall. Table \ref{table:dimensions} lists
key specifications of the \gls{MSR} model. Table \ref{table:properties} lists salt and graphite
material properties.

Salt flows upwards through the core and flows through a pump and a heat exchanger before
re-entering the core through a downcomer. The present work assumes a uniform gamma heat source in
the graphite regions cumulatively equal to 7.8~\% of total fission heat produced.

\begin{figure}[!h]
    \centering
    \begin{subfigure}[b]{0.369\textwidth}
      \centering
      \includegraphics[width=\textwidth]{3d-geometry}
      \caption{3-D \gls{MSR} geometry with a cutout}
      \label{fig:3d}
    \end{subfigure}
    \hfill
    \begin{subfigure}[b]{0.605\textwidth}
      \centering
      \includegraphics[width=\textwidth]{1d-geometry}
      \caption{1-D multichannel \gls{MSR} geometry}
      \label{fig:1d}
    \end{subfigure}
    \caption{3-D/1-D Moltres-\gls{SAM} \gls{MSR} geometries. The cutout shows the internal
      fuel channels in the 3-D model.}
    \label{fig:geometry}
\end{figure}

\subsection{SIMULATION SETUP}

Moltres models the 3-D core domain, while \gls{SAM} models the primary salt loop as a closed 1-D
flow loop, including 1-D pipes corresponding to each fuel channel and plena regions, as shown in
Figures \ref{fig:3d} and \ref{fig:1d}, respectively. We applied form loss coefficients derived by
\cite{faulkner_methodology_2025}. Out-of-core \gls{SAM} component specifications were taken from an
existing \gls{SAM} \gls{MSRE} model \cite{hu_fy21_2021} in the Virtual Test Bed Repository
\cite{giudicelli_virtual_2023}. The 3-D Moltres model consisted of 5,932,325 degrees of freedom on
362,070 2nd-order 3-D mesh elements. The 1-D \gls{SAM} model consisted of 87,431 degrees of freedom
on 4981 2nd-order 1-D mesh elements. Both meshes contained the same axial mesh resolution of 27
segments in the core region to ensure conservative data transfers.

Moltres applies monotone cubic interpolation on material
group constants generated using the OpenMC Monte Carlo particle transport simulations from 900~K to
1200~K at 100~K intervals. 

Ideally, the \gls{MSR} salt-graphite interfaces should have heat transfer coefficients that depend
on local state variables to account for developing flow and internal heat generation. However,
due to existing software constraints in \gls{SAM}, the current model uses a constant heat transfer
correlation for laminar flow ($Nu=4.36$).

The 1-D \gls{SAM} heat exchanger component is 2.5298-m long and contains a heat removal term of the
form $[h(T(x)-T_\text{s})]$ where $T(x)$ is the salt temperature and
$T_\text{s}=824.8167$~K. We derived $h$ analytically by defining the core inlet temperature
$T_\text{in}=908.15$~K and integrating across the 1-D heat exchanger region:

%\noindent
\begin{minipage}{0.45\textwidth}
  \begin{equation}
    T_\text{out} = T_\text{in}+\frac{Q}{\dot{m}c_p}
  \end{equation}
\end{minipage}
\hfill
\begin{minipage}{0.45\textwidth}
  \begin{equation}
    h = \frac{\dot{m}c_p}{AL}\ln\frac{T_\text{out}-T_\text{s}}{T_\text{in}-T_\text{s}}
  \end{equation}
\end{minipage}

where $\dot{m}$ is the mass flow rate, $c_p$ is the specific heat capacity, $A$ is the flow area,
$L$ is the length, and $T_\text{out}$ is the core outlet temperature.

We ran a pseudo-transient simulation towards steady state by having Moltres execute steady-state
$k$-eigenvalue neutron diffusion and heat conduction calculations while SAM-Lite executes
transient thermal-hydraulics and \gls{DNP} tracking calculations. Quarter-core power is
fixed at 2~MW, equivalent to 8~MW full-core power. We set SAM-Lite to run multiple
timesteps between steady-state Moltres calculations to accelerate the longest-lived \gls{DNP}
towards steady state. Timestep sizes for SAM-Lite started at 1 second and increased to 10 seconds
by the end of the simulation. We used the steady-state results to run a tightly coupled null
transient simulation using Picard iterations on 0.01-s timesteps to confirm that the system is at
steady state.

For a effective delayed neutron fraction ($\beta_\text{eff}$) loss estimate at steady state, we ran
another pseudo-transient simulation with stationary \glspl{DNP} to isolate the impact of \gls{DNP}
flow while retaining thermal-hydraulic feedback. The reactivity difference between the two
simulations provides the $\beta_\text{eff}$ loss estimate.

The simulations ran with 256 MPI ranks on two AMD EPYC 9754 128-core processors. \gls{SAM}-Lite ran
on 16 MPI ranks to avoid excessive parallelization and limit its memory usage from
replicated mesh storage. The Moltres subapp runs on a distributed memory mesh across all 256 MPI
ranks. At the start of the simulations, when the reactor state is rapidly changing,
each Moltres and SAM-Lite execution took approximately 1 minute and 0.5 minutes, respectively.
Towards the end near steady state, execution times dropped to approximately 0.5 minutes and 5
seconds, respectively. Data transfers took 7 seconds from \gls{SAM} to Moltres and less than 5
seconds from Moltres to \gls{SAM}.

\section{SIMULATION RESULTS \& DISCUSSION}

Figure \ref{fig:temperature} shows the steady-state temperature distributions in the 3-D graphite
region and the 1-D fuel salt loop. The salt temperature peaks at the central core channel outlets
as the salt advects heat out of the core. The two annular half-channels and two regular channels in
the center report cooler salt temperatures arising from higher flow rates according to MSRE
specifications. The graphite temperature peaks at just above the center of the
graphite region due to the salt being hotter at the top of the core. The peak graphite temperature
of 959~K is lower than the 976~K peak value computed by Haubenreich et al.\
\cite{haubenreich_msre_1964} from a 10~MW MSRE model with an inlet temperature of 910~K. Assuming
the temperature difference between the graphite central region and channel inlet scales linearly
with reactor power going from 8~MW to 10~MW, the peak graphite temperature would be 973~K and in
closer agreement with the reference value.

The temperature difference between the inlet and outlet temperatures in Table \ref{table:results}
is consistent with analytical energy balance calculations for 2~MW power within 0.007~\%.
The average salt circulation time of 25.1~s is also consistent with analytical flow calculations
from mass flow rates and loop geometry. These results verify that the model correctly conserves
mass and energy within the primary loop, confirming the proper implementation of the Moltres-SAM
coupling.

Since the \gls{MSR} model has the same in-core residence and total recirculation times as the
\gls{MSRE}, $\beta_\text{eff}$ loss due to \gls{DNP} drift should be comparable to
\gls{MSRE} simulation results in existing literature. In Table \ref{table:results}, the
$\beta_\text{eff}$ loss of 239~pcm agrees well with 242.8~pcm and 241~pcm losses reported by
Jaradat et al.\ \cite{jaradat_multiphysics_2023} and Shi \& Fratoni \cite{shi_gen-foam_2021}.
The axial \gls{DNP} production and decay distributions in Figure \ref{fig:precursor} agree
qualitatively \gls{DNP} distributions in the \gls{MSRE} benchmark report
\cite{fratoni_molten_2020} which reported 224.83~pcm loss. The unweighted $\beta$, calculated from
a simple ratio of delayed neutrons to total neutrons produced in the core, corresponds to an
unweighted $\beta$ loss of 130~pcm. The 46\% discrepancy between effective and unweighted
$\beta$ loss highlights the impact of \gls{DNP} drift and delayed neutron production in regions of
lower neutron importance in \glspl{MSR}.

\begin{figure}[!ht]
    \centering
    \includegraphics[width=0.6\columnwidth]{temperature-label}
    \caption{Graphite and fuel salt temperature distributions at steady-state.}
    \label{fig:temperature}
\end{figure}

\begin{figure}[!htb]
  \begin{minipage}{0.49\textwidth}
    \centering
    \footnotesize
    \captionof{table}{Steady-state reactor results.}
    \begin{tabular}{l c}
      \toprule
      Parameter & {Value} \\
      \midrule
      Inlet temperature [K] & 908.15 \\
      Outlet temperature [K] & 930.63 \\
      Maximum graphite temperature [K] & 959.02 \\
      Maximum salt temperature [K] & 943.52 \\
      Average salt circulation time [s] & 25.1 \\
      Multiplication factor, $k_\text{eff}$ [-] & 1.063684 \\
      Multiplication factor with static \gls{DNP}, $k_\text{eff}^\text{static}$ [-] & 1.066391 \\
      Effective delayed neutron fraction loss, $\beta_\text{eff}^\text{loss}$ [pcm] & 239 \\
      Unweighted delayed neutron fraction, $\beta$ [pcm] & 522 \\
      \bottomrule
    \end{tabular}
    \label{table:results}
  \end{minipage}
  \hfill
  \begin{minipage}{0.45\textwidth}
    \centering
    \includegraphics[width=\textwidth]{precursor}
    \caption{Total axial \gls{DNP} production and decay distributions from Moltres-\gls{SAM}.}
    \label{fig:precursor}
  \end{minipage}
\end{figure}

\section{CONCLUSIONS}

This work introduced a multiscale 3-D/1-D multiphysics simulation framework for channel-type
\gls{MSR} analysis by coupling Moltres to \gls{SAM} through native \gls{MOOSE} coupling interfaces.
The multiscale approach leverages the computational efficiency of 1-D system-level
thermal-hydralics while retaining much of the 3-D spatial resolution in neutron flux, \gls{DNP},
and temperature distributions.

We demonstrated and verified its capabilities through a steady-state simulation of a quarter-core
\gls{MSR} model derived heavily from the \gls{MSRE} design. The 1-D multichannel model represents
one of the largest \gls{SAM} models ever run based on the number of pipes. The steady-state results
agreed well with analytical energy balance calculations and reference \gls{MSRE} data. Further
verification work is ongoing to certify that the Moltres-\gls{SAM} coupling is consistent with
other multiphysics tools in modeling \glspl{MSR} through the \gls{MSR} progression
problem benchmarks \cite{gentry_graphite-moderated_2025}. Current efforts are also looking into
optimizing performance to reduce computational costs.

Future efforts include adding control rod modeling, performing transients that induce asymmetric
neutronics and thermal-hydraulics distributions and, exploring how this multiphysics tool can fit
into a digital twin framework for data-driven model training and periodic calibration during reactor operation.

\section*{ACKNOWLEDGEMENTS}
{%\small
This work was supported by the State of Texas through the Digital Molten Salt Reactor Initiative.
The authors acknowledge the Texas Advanced Computing Center (TACC) at The University of Texas at
Austin for providing computational resources that have contributed to the research results reported
within this paper. The authors thank the SAM development team at Argonne National Laboratory
for their technical support.
}

{%\small
\bibliographystyle{physor2026}
\bibliography{main}
}

%\appendix
%
%\section{}
%If necessary, include Appendices numbered in upper case alphabetical order.
%
%To ensure a uniform, professional look at the proceedings, please only modify the format of this template after checking with the publication chair first.


\end{document}
